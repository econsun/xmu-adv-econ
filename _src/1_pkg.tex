% cSpell: disable

% -----------------------------------------------
% -----------------------------------------------
% 字体 字体 字体 字体 字体 字体 字体 字体 字体 字体 字体
% -----------------------------------------------
% -----------------------------------------------

\usepackage[T1]{fontenc}
\usepackage[theoremfont]{newpxtext}
\usepackage[vvarbb]{newpxmath}
\usepackage[scr=boondoxo]{mathalfa}

% -----------------------------------------------
% -----------------------------------------------
% 排版 排版 排版 排版 排版 排版 排版 排版 排版 排版 排版
% -----------------------------------------------
% -----------------------------------------------

\usepackage{geometry}
\usepackage{microtype}

% 页面边距
\geometry{left=1in, right=1in, top=1in, bottom=1in}

% 行间距
\linespread{1.3}
\setlength{\lineskiplimit}{1.2pt}
% 段间距
\setlength{\parskip}{0.7em}

% 首行缩进
\usepackage{indentfirst}
\setlength{\parindent}{2em}

% 标题 section 上下间距
% 三个参数:左间距、上间距、下间距
\usepackage{titlesec}
\titlespacing*{\section}{0pt}{16pt}{8pt}

% 免报告
\hbadness=10000

% -----------------------------------------------
% -----------------------------------------------
% 目录 目录 目录 目录 目录 目录 目录 目录 目录 目录 目录
% -----------------------------------------------
% -----------------------------------------------

\usepackage{tocloft}
\usepackage{multicol}

% -----------------------------------------------
% -----------------------------------------------
% 页眉 页脚 页眉 页脚 页眉 页脚 页眉 页脚 页眉 页脚 页眉
% -----------------------------------------------
% -----------------------------------------------

\usepackage{fancyhdr}

\pagestyle{fancy}
\fancyhf{}

\lhead{\courseabbname}           % 左 页眉
\chead{\assignmentname}          % 中 页眉 
\rhead{\myname}                  % 右 页眉
\cfoot{\thepage}                 % 中 页脚

\fancyheadoffset{0pt}            % 确保页眉宽度为文本宽度
\setlength{\headheight}{14pt}    % 设置页眉高度,防止警告

% -----------------------------------------------
% -----------------------------------------------
% 颜色 颜色 颜色 颜色 颜色 颜色 颜色 颜色 颜色 颜色 颜色
% -----------------------------------------------
% -----------------------------------------------

\usepackage{color}
\usepackage{xcolor}

% 定义颜色,请确保使用时颜色名字无误

\definecolor{xmublue}     {HTML}{003C88}
\definecolor{xmured}      {HTML}{D7422A}

\definecolor{myred}       {HTML}{F4A8B0}
\definecolor{mypink}      {HTML}{E5B1B5}
\definecolor{myorange}    {HTML}{FFC187}
\definecolor{myyellow}    {HTML}{FFF29A}
\definecolor{mygreen}     {HTML}{B8C49A}
\definecolor{myblue}      {HTML}{A5BAD4}
\definecolor{mypurple}    {HTML}{C2B7D5}
\definecolor{mygray}      {HTML}{EFEFEF}
\definecolor{mywhite}     {HTML}{FFFFFF}

\definecolor{linkblue}    {HTML}{0066CC}
\definecolor{linkred}     {HTML}{E7F5D7}
\definecolor{linkgreen}   {HTML}{00994C}
\definecolor{linkorange}  {HTML}{FF6600}

% -----------------------------------------------
% -----------------------------------------------
% 链接 链接 链接 链接 链接 链接 链接 链接 链接 链接 链接
% -----------------------------------------------
% -----------------------------------------------

\usepackage[colorlinks=true,hyperfootnotes=true]{hyperref}

% 避免 带 * 标题 冲突而无法引用
\makeatletter
\newcommand\setlabel[1]{\def\@currentlabelname{#1}\label{#1}}
\makeatother

% 设置 链接 颜色
\hypersetup{
    linkcolor=linkblue,    % 链接 颜色
    citecolor=linkgreen,   % 引用 颜色
    urlcolor=linkblue    % 网址 颜色
}

% -----------------------------------------------
% -----------------------------------------------
% 脚注 脚注 脚注 脚注 脚注 脚注 脚注 脚注 脚注 脚注 脚注
% -----------------------------------------------
% -----------------------------------------------

\usepackage[bottom,hang,flushmargin]{footmisc}

% 脚注区和正文安全距离
\renewcommand{\footnoterule}{%
  \kern 10pt % 与正文之间的初始间距,可以调节
  \hrule width 0.4\textwidth height 0.4pt % 横线宽度和高度
}

% 正文区 脚注符号 左右间距
\makeatletter
\renewcommand{\@makefnmark}{\hbox{\hspace{2pt}\@textsuperscript{\normalfont[\@thefnmark]}\hspace{2pt}}}
\makeatother

\setlength{\footnotemargin}{1.2em}  % 脚注区 符号 和 内容 间距
\setlength{\footnotesep}{14pt}      % 脚注区 不同脚注 间距

% -----------------------------------------------
% -----------------------------------------------
% 列表 列表 列表 列表 列表 列表 列表 列表 列表 列表 列表
% -----------------------------------------------
% -----------------------------------------------

\usepackage{enumitem}

\setlist[enumerate,1]{label=(\alph*), left=1.5em, labelsep=0.5em}

\newlist{cirlist}{enumerate}{1}
\setlist[cirlist,1]{label={\raisebox{0.2ex}{\scalebox{0.7}{$\bullet$}}}, left=1.5em, labelsep=0.5em}

\newlist{numlist}{enumerate}{1}
\setlist[numlist,1]{label=\arabic*., left=0.5em, labelsep=0.5em}

% -----------------------------------------------
% -----------------------------------------------
% 图片 图片 图片 图片 图片 图片 图片 图片 图片 图片 图片
% -----------------------------------------------
% -----------------------------------------------

\usepackage{graphicx}
\usepackage{caption}
\usepackage{float}

\setlength{\intextsep}{8pt}          % 图片 上下间距
\setlength{\abovecaptionskip}{4pt}   % 标签 上间距
\setlength{\belowcaptionskip}{0pt}   % 标签 下间距

\usepackage{tikz}           % 绘图
\usepackage{pgfplots}       % 坐标系和绘图更方便
\pgfplotsset{compat=1.18}   % 兼容版本设置

% -----------------------------------------------
% -----------------------------------------------
% 表格 表格 表格 表格 表格 表格 表格 表格 表格 表格 表格
% -----------------------------------------------
% -----------------------------------------------

\usepackage{array}
\usepackage{booktabs}
\usepackage{longtable}
\usepackage{threeparttable}
\usepackage{threeparttablex}
\usepackage{makecell}

% -----------------------------------------------
% -----------------------------------------------
% 数学 数学 数学 数学 数学 数学 数学 数学 数学 数学 数学
% -----------------------------------------------
% -----------------------------------------------

\let\openbox\relax
\let\Bbbk\relax

\usepackage{amsmath}
\usepackage{amssymb}
\usepackage{amsthm}
\usepackage{mathtools}
\usepackage{bm}
\usepackage{mathrsfs}
\usepackage{dsfont}
\usepackage{cases}
\usepackage{empheq}
\usepackage{extarrows}

\usepackage{xparse}

\setlength{\jot}{6pt}
\setlength{\arraycolsep}{4pt}

\AtBeginDocument{%
  \setlength{\abovedisplayskip}{8pt}
  \setlength{\belowdisplayskip}{8pt}
  \setlength{\abovedisplayshortskip}{8pt}
  \setlength{\belowdisplayshortskip}{8pt}
}

\newcommand{\mathspace}{\hspace{0.12em}}
\let\oldinline\(
\renewcommand{\(}{\oldinline\mathspace}
\let\oldendinline\)
\renewcommand{\)}{\mathspace\oldendinline}

% -----------------------------------------------
% -----------------------------------------------
% 其他 其他 其他 其他 其他 其他 其他 其他 其他 其他 其他
% -----------------------------------------------
% -----------------------------------------------

% 忽略所有警告
\usepackage{silence}
\WarningsOff*