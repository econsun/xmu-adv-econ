% 更灵活设置命令
\usepackage{xparse}
% 大写首字母命令
\usepackage{mfirstuc}

% 文字类宏

\newcommand{\hwnumber}[1]{Assignment \;\#\;#1}
\newcommand{\hint}{\par\fafugreen{\textsc{\bf Hint}}\,: }
\newcommand{\remark}{\par\fafugreen{\textsc{\bf Remark}}\,: }
\newcommand{\vhint}{\par\vspace{0.7em}\fafugreen{\textsc{\bf Hint}}\,: }
\newcommand{\vremark}{\par\vspace{0.7em}\fafugreen{\textsc{\bf Remark}}\,: }
\newcommand{\past}[1]{{\textcolor{xmublue}{\bf(#1)}} }
\newcommand{\score}[1]{{\textcolor{xmublue}{\bf(#1 points)}} }

\newcommand{\xmured}[1]{{\color{xmured}#1}}
\newcommand{\xmublue}[1]{{\color{xmublue}#1}}
\newcommand{\fafugreen}[1]{{\color{fafugreen}#1}}

% 数学类宏

\let\originalthinspace\, 
\renewcommand{\,}{\ifmmode \mspace{2mu} \else \originalthinspace \fi}

\newcommand{\spacexs}{\mspace{1mu}}
\newcommand{\spaces}{\mspace{2mu}}
\newcommand{\spacem}{\mspace{4mu}}
\newcommand{\spacel}{\mspace{6mu}}

\renewcommand{\d}{\mathrm{d}\spacexs }
\newcommand{\p}{\partial\spaces }

\newcommand{\pp}[1]{\dfrac{\partial}{\p #1}}
\newcommand{\pfkal}{\p F\bka{K,AL}}
\newcommand{\pfkalpl}{\p F\bka{K,AL}/\,\p L}
\newcommand{\pfkalpk}{\p F\bka{K,AL}/\,\p K}

\newcommand{\matspace}{\mspace{4mu}}

\newcommand{\islash}{{\spaces}/{\spaces}}
\newcommand{\bigslash}{{\spaces}\big/{\spaces}}
\newcommand{\biggslash}{{\spaces}\bigg/{\spaces}}
\newcommand{\Bigslash}{{\spaces}\Big/{\spaces}}
\newcommand{\Biggslash}{{\spaces}\Bigg/{\spaces}}

\newcommand{\ivbar}{{\spaces}|{\spaces}}
\newcommand{\bigbar}{{\spaces}\big|{\spaces}}
\newcommand{\biggbar}{{\spaces}\bigg|{\spaces}}
\newcommand{\Bigbar}{{\spaces}\Big|{\spaces}}
\newcommand{\Biggbar}{{\spaces}\Bigg|{\spaces}}

\newcommand{\mean}{\text{E}}
\newcommand{\meana}[1]{\text{E}\bka{#1}}
\newcommand{\means}[1]{\text{E}\bks{#1}}
\newcommand{\meand}[1]{\text{E}\bkd{#1}}
\newcommand{\meanf}[1]{\text{E}\bkf{#1}}

\newcommand{\var}{\text{Var}}
\newcommand{\vara}[1]{\text{Var}\bka{#1}}
\newcommand{\vars}[1]{\text{Var}\bks{#1}}
\newcommand{\vard}[1]{\text{Var}\bkd{#1}}
\newcommand{\varf}[1]{\text{Var}\bkf{#1}}

\newcommand{\pr}{\Pr }
\newcommand{\prA}{\Pr\bka{A}}
\newcommand{\prB}{\Pr\bka{B}}
\newcommand{\prC}{\Pr\bka{C}}
\newcommand{\pra}[1]{\Pr\bka{#1}}
\newcommand{\prs}[1]{\Pr\bks{#1}}
\newcommand{\prd}[1]{\Pr\bkd{#1}}



\newcommand{\bka}[1]{\left(\spacexs#1\spacexs\right)}
\newcommand{\bks}[1]{\left[\spacexs#1\spacexs\right]}
\newcommand{\bkd}[1]{\left\{\spacexs#1\spacexs\right\}}
\newcommand{\bkf}[1]{\left|\spacexs#1\spacexs\right|}
\newcommand{\bkg}[1]{\left\|\spacexs#1\spacexs\right\|}
\newcommand{\bkas}[1]{\left(\spacexs#1\spacexs\right]}
\newcommand{\bksa}[1]{\left[\spacexs#1\spacexs\right)}

\newcommand{\matbka}[1]{\left(\matspace\begin{matrix}#1\end{matrix}\matspace\right)}
\newcommand{\matbks}[1]{\left[\matspace\begin{matrix}#1\end{matrix}\matspace\right]}
\newcommand{\matbkf}[1]{\left|\matspace\begin{matrix}#1\end{matrix}\matspace\right|}
\newcommand{\matbkv}[1]{[\spaces\begin{matrix}#1\end{matrix}\spaces]}

\newcommand{\bkt}{\bka{t}}
\newcommand{\bkk}{\bka{k}}
\newcommand{\bkz}{\bka{z}}
\newcommand{\bkA}{\bka{A}}
\newcommand{\bkkss}{\bka{k^*}}
\newcommand{\bkkl}{\bka{K,L}}
\newcommand{\bkkal}{\bka{K,AL}}
\newcommand{\bkx}{\bka{x}}
\newcommand{\bkxx}{\bka{x_1,x_2}}
\newcommand{\bkxy}{\bka{x,y}}
\newcommand{\bkbsx}{\bka{\boldsymbol{x}}}
\newcommand{\bkbspw}{\bka{\boldsymbol{p},w}}
\newcommand{\bkbspu}{\bka{\boldsymbol{p},u}}

\newcommand{\bkdxn}{\bkd{x_n}}
\newcommand{\bkdyn}{\bkd{y_n}}


\newcommand{\bs}[1]{\boldsymbol{#1}}
\newcommand{\bbs}[1]{\bar{\boldsymbol{#1}}}
\newcommand{\hbs}[1]{\hat{\boldsymbol{#1}}}
\newcommand{\tbs}[1]{\tilde{\boldsymbol{#1}}}

\newcommand{\bsx}{\boldsymbol{x}}
\newcommand{\bsp}{\boldsymbol{p}}
\newcommand{\bsy}{\boldsymbol{y}}
\newcommand{\bsz}{\boldsymbol{z}}
\newcommand{\bsA}{\boldsymbol{A}}

\newcommand{\call}{\mathcal{L}}



\newcommand{\argmax}{\mathop{\text{argmax}}}
\newcommand{\argmin}{\mathop{\text{argmin}}}


\newcommand{\bbrn}{\bbr[n]}
\NewDocumentCommand{\bbr}{o}{%
  \mathbb{R}\IfValueT{#1}{^{#1}}%
}

\NewDocumentCommand{\bb}{m o}{%
  \mathbb{#1}\IfValueT{#2}{^{#2}}%
}

\RenewDocumentCommand{\cal}{m o}{%
  \mathcal{#1}\IfValueT{#2}{^{#2}}%
}

\NewDocumentCommand{\scr}{m o}{%
  \mathscr{#1}\IfValueT{#2}{^{#2}}%
}

\newcommand{\calb}{\mathcal{B}}
\renewcommand{\call}{\mathcal{L}}

\newcommand{\te}[1]{\text{#1}}
\newcommand{\teu}[1]{\text{\MakeUppercase{#1}}}
\newcommand{\noun}[1]{\text{#1}}

\newcommand{\inv}{^{-1}}
\newcommand{\tra}{^{T}}
\newcommand{\abfun}{\,\cdot\,}
\newcommand{\dis}{\displaystyle}

\newcommand{\intpiz}{\int^{\infty}_{0}}

\newcommand{\parspace}{\vspace{0.7em}}

\newcommand{\poisson}{\te{Poisson}}
\newcommand{\normal}{\te{Normal}}
\newcommand{\exponential}{\te{Exponential}}
\newcommand{\bernoulli}{\te{Bernoulli}}
\newcommand{\binomial}{\te{Binomial}}
\newcommand{\geometric}{\te{Geometric}}
\newcommand{\uniform}{\te{Uniform}}
\newcommand{\gammaf}{\te{Gamma}}
\newcommand{\betaD}{\te{Beta}}

\newcommand{\ninn}{N\in\mathbb N}

\NewDocumentCommand{\solowaxis}{}{
    \draw (0,0) -- (8,0) node[below] {\(k\)};
    \draw (0,0) -- (0,5) node[above left, rotate=90, align=right] {Output and investment\\per unit of effective labor};
}

\NewDocumentCommand{\solowproductfunction}{m O{}}{
    \draw[#2] (0,0) 
        .. controls (0.2,2.4) and (4,4) .. (7,4.4) 
        node[right] {#1};
}

\NewDocumentCommand{\solowinvestfunction}{m O{}}{
    \draw[#2] (0,0) 
        .. controls (0.5,1.2) and (3.8,2.1) .. (7,2.3) 
        node[right] {#1};
}

\NewDocumentCommand{\solowbreakeven}{m m O{}}{%
    \draw[#3] (0,0) -- (#1) node[right] {#2};
}

\NewDocumentCommand{\solowequilibrium}{m m m O{} O{}}{%
    \filldraw[#4] (#1,#2) circle (1.5pt);
    \draw[#5] (#1,#2) -- (#1,0) node[below] {#3};
}

\NewDocumentCommand{\solowarrow}{m m m m}{%
    \draw[->] (#1,#2) -- ++(#3:#4);
}

\newcommand{\grt}[1]{\dfrac{\dot{#1}}{#1}}
\newcommand{\grta}[1]{\dfrac{\dot{\bka{#1}}}{#1}}
\newcommand{\grts}[1]{\dfrac{\dot{\bks{#1}}}{#1}}
\newcommand{\ingrt}[1]{\dot{#1}\islash{#1}}
\newcommand{\ingrta}[1]{\dot{\bka{#1}}\islash{#1}}
\newcommand{\ingrts}[1]{\dot{\bks{#1}}\islash{#1}}

\newcommand{\notation}[1]{(\(#1\))}

\renewcommand{\and}{\quad\te{and}\quad}
\renewcommand{\implies}{\quad\Rightarrow\quad}